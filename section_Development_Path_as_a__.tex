\section{ Development Path as a Teacher, Strengths, Weaknesses, Opportunities and Challenges, Teacher Idententy and Direction of Development}

I feel similarities in my teacher development path and identity as with the ``Case Anne'' as described by Juuti and Raehalme \cite{juuti_2013}. 
As Anne, it has been a long time dream for me to become a teacher. 
Already when I was a kid, since I had ease with math and physics classes and enjoyed to help my classmate, I envisioned to become a math teacher.
And as Anne, when I started teaching, while I was feeling confident with the teaching subjects, I was also spending very much time in preparing the material and spent my weekends and evenings in studying all the content of my courses very deeply. 


As Beijaard et al \cite{Beijaard_2004} and Nissila \cite{nissila_2013} state, ``the teacher identity is dynamic, always becoming and should not be retrospective''. 
While I still see it as a form of respect for my students that I master the topics to guarantee them quality teaching, I quickly lost the fear to answer them ``I don't know'' when I do not have the answer. 
With the time, I also started to use humor and jokes during my teaching and I also started to be more relax and sometimes less strict (e.g. giving them extra deadline, offering the possibility to redo a project to improve their grade while the rules state that only an exam could be redone,\ldots). 

From another perspective, I would have difficulties to describe my transition from an expert to a vocational teacher as described by Juuti and Raehalme \cite{juuti_2013}. From the beginning of my career I was teaching and working in parallel. And I hope I will continue to live in these two worlds together.

  
  
  
  
  
  
  
  