\section{Personal Development and Educational Challenges in the Teacher's pedagogical studies}

While the previous section should have describe my weaknesses as a vocational teacher, I would prefer to describe and use them as challenges to improve myself.
I had and will still have different teaching contexts. 
One is first year mandatory course that every student must pass.
This type of course have students ranging from absolute beginner to experts.
It also add to the mix a wide variety of motivation: those eager to learn, those who knows but are happy to deepen their knowledge and those who do not care at all but just need the credits.
In that case, challenges are:
\begin{itemize}
\item how can I balance the teaching to have simple enough content for the beginners to get started and advanced topics to keep challenging the experts?
\item how could I motivate the experts students to tutor the beginners without forcing them to do so?
\item how could I prevent sabotage from those who do not care, in order to keep a good learning atmosphere?
\item what are the tools to get the students engaged, interactive and participative very early in the course? While I get it after 2-3 lectures, at the beginning they are frequently a bit shy/afraid. Even when go and ask individually during the labs/exercises hours, there will still be some that tell me ``no problem'' and then I will only notice few lectures later that they are totally lost and lagging behind, which makes it harder for me and them to catch up. 
\end{itemize}
  
In the others type of courses, 