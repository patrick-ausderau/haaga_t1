After high-school, I try mathematics at university. While the subject was and still is of great interest to me, the amount of work and the rhythm was too fast for me. 
To stay in the scientific/technical world but with more hands-on practice, I made an accelerated apprenticeship in computer science and became an IT guy. 
I then moved to a university of applied sciences where I continued to study computer science and got a bachelor in software engineering while working in parallel in a small sport import company providing them with helpdesk support.

In 2005, at the end of my studies, I moved to Finland where I started to work as a research engineer in Metropolia University of Applied Sciences (called Evtek at that time). 
Very quickly, I received the opportunity to teach software related topics such as web technologies, programming languages, software engineering and data management in parallel to my research tasks. 
After completing my master degree in mobile programming last year and with the hope of getting a lecturer position, vocational studies became a must.

For ten years now, I have been working for many EU projects with many related to pedagogy. 
In 2005, I started by maintaining the NetPro application and server which was a online learning environment, similar to e.g. moodle, where teacher could upload teaching material, create assignment for students, assess them and grade the students,\ldots
From 2006 to 2011, I participated in the KP-Lab\footnote{Knowledge Practices Laboratory (\url{http://www.kp-lab.org/})} where \citeauthor{Paavola_2005} \cite{Paavola_2005} developed the Trialogical Learning metaphor\footnote{see also \url{http://kplab.evtek.fi:8080/wiki/Wiki.jsp?page=TrialogicalLearning}}.





  
  
  